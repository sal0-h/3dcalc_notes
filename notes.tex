\documentclass{report}
\usepackage{hyperref} % For clickable links and bookmarks
\usepackage{bookmark}

\usepackage[utf8]{inputenc}
\usepackage{enumitem}
\usepackage{amsmath} % For math environments
\usepackage{tcolorbox}
\usepackage{romannum}
\input{preamble}
\setlength{\parindent}{0pt} % Ensure that no indent is added
\title{21-259 Calculus in 3D}
\date{Midterm 2 Edition}
\author{Compiled by Salman Hajizada}

\begin{document}
\maketitle

\setcounter{chapter}{-1}
\chapter{Disclaimer}
This is not a textbook. Do not use this text to learn the concepts.
This is a revision guide or notes used to refresh your memory of the theorems. 
The best way to learn the content is to read the book, work through
the guided exercises and do some practice problems. 

Good luck,

SH
\setcounter{chapter}{11}
\chapter{Vectors and the Geometry of Space}

\section{Three-Dimensional Coordinate System}

\textbf{Right-hand rule} for determining the coordinate axis:
\begin{center}
    \includegraphics[width=0.5\textwidth]{images/right_hand_rule.png}
\end{center}

Projection of a point $(x, y, z)$ onto
\begin{itemize}
    \item $x-y$ plane: (x, y, 0)
    \item $x-z$ plane: (x, 0, z)
    \item $y-z$ plane: (0, y, z)
\end{itemize} 

\textbf{Distance formula in 3 dimensions:} Between two points $(x_1, y_1, z_1)$ and 
$(x_2, y_2, z_2)$, the distance is given by:
\[\sqrt{(x_2 - x_1)^2 + (y_2 - y_1)^2 +(z_2 - z_1)^2}\]

\textbf{Equation of a Sphere} with center at $(h, k, l)$ and radius $r$ is:

\[(x - h)^2 + (y - k)^2 + (z - l)^2 = r^2\]

\section{Vectors}

I will skip the basics, like vector addition definition, etc.

Components of a vector (for notation):
\[\aaa = \langle a_1, a_2, a_3 \rangle\]

Given points $A(x_1, y_1, z_1)$ and $B(x_2, y_2, z_2)$, 

\[\overrightarrow{AB} = \aaa = \langle x_2 - x_1, y_2 - y_1, z_2 - z_1 \rangle\]

\textbf{Magnitude of a vector} $\samplevec$ is 
\[|\aaa| = \sqrt{a_1^2 + a_2^2 + a_3^2}\]

\textbf{Unit vectors:} 
\[\ii = \langle 1, 0, 0 \rangle, \quad \jj = \langle 0, 1, 0 \rangle, \quad \kk = \langle 0, 0, 1 \rangle\]


\begin{tcolorbox}[colback=blue!5!white, colframe=blue!75!black, title=Algebraic Properties of Vectors in $\Rn$]
$\forall$ $\uu, \vv, \ww \in \Rn$ and scalars $c$ and $d$:
\begin{enumerate}
    \item \textbf{Commutativity of addition:} $\uu + \vv = \vv + \uu$
    \item \textbf{Associativity of addition:} $\uu + (\vv + \ww) = (\uu + \vv) + \ww$
    \item \textbf{Additive identity:} $\uu + \0 = \uu$
    \item \textbf{Additive inverse:} $\uu + (-\uu) = \0$
    \item \textbf{Distributivity of scalar multiplication over addition (vectors):} $c(\uu + \vv) = c\uu + c\vv$
    \item \textbf{Distributivity of scalar multiplication over addition (scalars):} $(c + d)\uu = c\uu + d\uu$
    \item \textbf{Associativity of scalar multiplication:} $(cd)\uu = c(d\uu)$
    \item \textbf{Scalar identity:} $1 \uu = \uu$
\end{enumerate}
\end{tcolorbox}

\section{The Dot Product}

\begin{definition}
If $\samplevec$ and $\bb = \langle b_1, b_2, b_3\rangle$ then the \textbf{dot product}
of $\aaa $ and $\bb$ is 
\[
\aaa \cdot \bb = a_1 b_1 + a_2 b_2 + a_3 b_3
\]
\end{definition}

\begin{tcolorbox}[colback=red!5!white, colframe=red!75!black, title=Properties of the Dot Product]
$\forall$ vectors $\aaa, \bb, \cc$ and scalar $c$:
\begin{enumerate}
    \item $\aaa \cdot \aaa = |\aaa|^2$
    \item $\aaa \cdot \bb = \bb \cdot \aaa$
    \item $\aaa \cdot (\bb + \cc) = \aaa \cdot \bb + \aaa \cdot \cc$
    \item $(c \aaa) \cdot \bb = c (\aaa \cdot \bb) = \aaa \cdot (c \bb)$
    \item $\0 \cdot \aaa = 0$
\end{enumerate}
\end{tcolorbox}

\begin{theorem}
    If $\theta$ is the angle between two vectors $\aaa, \bb$ then 
    \[\aaa \cdot \bb = |\aaa| |\bb| \cos \theta\]
\end{theorem}

\begin{corollary}
    Two vectors $\aaa, \bb$ are orthogonal iff $\aaa \cdot \bb = 0$
\end{corollary}

The \textbf{direction angles} of a non-zero vector $\aaa$ are the angles 
$\alpha, \beta, \gamma$ that $\aaa$ makes with the positive $x$-, $y$-, and $z$-axes, respectively.

The cosines of these angles are called \textbf{direction cosines}:

\[\cos \alpha = \frac{\aaa \cdot \ii}{|\aaa||\ii|} = \frac{a_1}{|\aaa|},
\quad \cos \beta = \frac{a_2}{|\aaa|},
\quad \cos \gamma = \frac{a_3}{|\aaa|}\]

Some nice properties of direction cosines:

\[\cos^2 \alpha + \cos^2 \beta + \cos^2 \gamma = 1\]

\[\frac{\aaa}{|\aaa|} = \langle \cos \alpha, \cos \beta, \cos \gamma \rangle\]

\textbf{Projections:}

The \textbf{projection of} $\bb$ \textbf{onto} $\aaa$ means we drop a perpendicular 
from the tip of $\bb$ onto the line spanned by $\aaa$. The resulting vector is called 
the vector projection of $\bb$ onto $\aaa$.
\\

The \textbf{scalar projection of} $\bb$ \textbf{onto} $\aaa$ is the length of this 
projection, given by
\[
\text{comp}_{\aaa} \bb = |\bb|\cos\theta = \frac{\aaa \cdot \bb}{|\aaa|},
\]
where $\theta$ is the angle between $\aaa$ and $\bb$.
\\

The \textbf{vector projection} of $\bb$ onto $\aaa$ is obtained by multiplying the 
unit vector in the direction of $\aaa$ by the scalar projection:
\[
\text{proj}_{\aaa} \bb = \left(\text{comp}_{\aaa} \bb \right)\frac{\aaa}{|\aaa|} 
= \frac{\aaa \cdot \bb}{|\aaa|^2}\aaa
\]

\section{The Cross Product}

\begin{definition}
For $\samplevec$ and $\bb = \langle b_1, b_2, b_3 \rangle$, the \textbf{cross product}
of $\aaa$ an $\bb$ is:
\[\aaa \times \bb = \langle a_2 b_3 - a_3 b_2, a_3 b_1 - a_1 b_3, a_1 b_2 - a_2 b_1 \rangle = \begin{vmatrix}
\mathbf{i} & \mathbf{j} & \mathbf{k} \\
a_1 & a_2 & a_3 \\
b_1 & b_2 & b_3
\end{vmatrix}
\]
\end{definition}

\begin{theorem}
    The vector $\aaa \times \bb$ is orthogonal to both $\aaa$ and $\bb$
\end{theorem}

\begin{theorem}
    The length of the cross product is given by: 
    \[ |\aaa \times \bb| = |\aaa| |\bb| \sin \theta\]
    where $\theta$is the angle between $\aaa$ and $\bb$.
\end{theorem}

\begin{corollary}
    Two nonzeros vectors $\aaa$ and $\bb$ are parallel iff 
    \[\aaa \times \bb = \0\]
\end{corollary}

\textbf{Area of parallelogram} determined by $\aaa$ and $\bb$ is $|\aaa \times \bb|$
\\

\textbf{Area of triangle} determined by $\aaa$ and $\bb$ is $\frac{1}{2}|\aaa \times \bb|$
\\

\textbf{IMPORTANT:} cross product is NOT commutative, and associative law for multiplication does not usually hold.

\begin{tcolorbox}[colback=red!5!white, colframe=red!75!black, title=Properties of the Cross Product]
$\forall$ vectors $\aaa, \bb, \cc$ and scalar $c$:
\begin{enumerate}
    \item $\aaa \times \bb = - \bb \times \aaa$
    \item $(c \aaa) \times \bb = c(\aaa \times \bb) = \aaa \times (c \bb)$
    \item $\aaa \times (\bb + \cc) = \aaa \times \bb + \aaa \times \cc$
    \item $(\aaa + \bb) \times \cc = \aaa \times \cc + \bb \times \cc$
    \item $\aaa \cdot (\bb \times \cc) = (\aaa \times \bb) \cdot \cc$
    \item $\aaa \times (\bb \times \cc) = (\aaa \cdot \cc) \bb - (\aaa \cdot \bb) \cc$
\end{enumerate}
\end{tcolorbox}

\textbf{Triple products:}

\[\aaa \cdot (\bb \times \cc) = \begin{vmatrix}
    a_1 & a_2 & a_3 \\
    b_1 & b_2 & b_3 \\
    c_1 & c_2 & c_3 \\
\end{vmatrix}\]


\textbf{Volume of the parallelepiped} determined by the vectors $\aaa, \bb$ and $\cc$
is:
\[V = |\aaa \cdot (\bb \times \cc)|\]

Useful: if the volume is 0, then the vectors must lie in the same plane, i.e. \textbf{coplanar}

\section{Equations of Lines and Planes}

\subsection{Equation of Lines}

\textbf{The equation of a line} that goes through point with position vector $\rr_0$, 
and is parallel to vector $\vv$ is \[
\rr = \rr_0 + t \vv
\]

If $\vv = \langle a, b, c\rangle$, $\rr_0 = \langle x_0, y_0, z_0 \rangle$ then 
\[
\langle x, y, z\rangle =  \langle x_0 + t a, y_0 + t b, z_0 + t c\rangle
\].

\textbf{Parametric form:}
\[
x = x_0 + t a \quad y = y_0 + t b \quad z = z_0 + t c
\]

\textbf{Direction numbers} of a line L are the numbers $a, b, c$ from above. 
\\

\textbf{Symmetric form:} (eliminating $t$) 
\[
\frac{x - x_0}{a} = \frac{y - y_0}{b} = \frac{z - z_0}{c}
\]

IMPORTANT: If $a, b$ or $c$ is zero, then we cannot divide by them.
For example, if $a = 0$ then the equation becomes:

\[
x = x_0, \quad \frac{y - y_0}{b} = \frac{z - z_0}{c}
\]

In general, the symmetric equations between two points $P_0(x_0, y_0, z_0), P_1(x_1, y_1, z_1)$ are 
\[
\frac{x - x_0}{x_1 - x_0} = \frac{y - y_0}{y_1 - y_0} = \frac{z - z_0}{z_1 - z_0}
\]

\textbf{The line segment} from $\rr_0$ to $\rr_1$ is given by the vector equation:
\[
\rr(t) = (1 - t)\rr_0 + t \rr_1 \quad \quad 0 \le t \le 1
\]

\subsection{Equation of Planes}

Let: 
\begin{itemize}
    \item $\mathbf{n} = \langle a, b, c \rangle$: vector orthogonal to plane
    \item $\rr_0 = \langle x_0, y_0, z_0 \rangle$: a position vector for a point in the plane 
    \item $\rr = \langle x, y, z \rangle$: a position vector for an arbitrary point in the plane.
\end{itemize}

Then the \textbf{vector equation of the plane} is: 
\[
\mathbf{n} \cdot (\rr - \rr_0) = 0 \quad \text{or} \quad \mathbf{n} \cdot \rr = \mathbf{n} \cdot \rr_0
\]

The \textbf{scalar equation} is:
\[
a(x - x_0) + b(y - y_0) + c(z - z_0) = 0
\]

The \textbf{linear equation} is:
\[
ax + by + cz - (a x_0 + b y_0 + c z_0) = 0
\] 

Two planes are parallel if their normal vectors are parallel.

If two planes are not parallel, then they intersect in a straight line and the angle between the two planes 
is defined as the acute angle between their normal vectors.

The \textbf{distance} $D$ from point $(x_1, y_1, z_1)$ to the plane 
$ax + by + cz + d = 0$ is 
\[
D = \frac{|a x_1 + b y_1 + c z_1 + d|}{\sqrt{a^2 + b^2 + c^2}}
\]

\chapter{Vector Functions}

Not coming

\chapter{Partial Derivatives}

\section{Functions of Several Variables}
Not coming in midterm 1 $:)$

\section{Limits and Continuity}
Not coming in midterm 1 $:)$

\section{Partial Derivatives}

If $f$ is a function of two variables, its \textbf{partial derivatives} are the functions 
$f_x, f_y$ defined by \[
f_x(x, y) = \lim_{h \to 0} \frac{f(x+h, y) - f(x, y)}{h}\]
and \[f_y(x, y) = \lim_{h \to 0} \frac{f(x, y+h) - f(x, y)}{h}\]

Also written: 
\[
f_x(x, y) = \frac{\partial f}{\partial x} = \frac{\partial z}{\partial x}
\]

\textbf{Rules for finding the partial derivatives of } $z = f(x, y)$
\begin{itemize}
    \item To find $\parx$ regard $y$ as a constant and differentiate $f(x, y)$ with respect to $x$
    \item To find $\pary$ regard $x$ as a constant and differentiate $f(x, y)$ with respect to $y$
\end{itemize}

\textbf{Interpretation of partial derivatives:}

The partial derivatives represent the instantaneous rate of change of the function's output with respect to one variable, while all other variables are held constant.
\begin{itemize}
    \item $f_x(x,y)$ is the rate at which $f$ changes with respect to $x$ when $y$ is fixed.
    \item $f_y(x,y)$ is the rate at which $f$ changes with respect to $y$ when $x$ is fixed.
\end{itemize}

\textbf{Higher Derivatives}: we can take second partial derivatives of $f$. Here is the notation:
\begin{align*}
(f_x)_x = f_{xx} &= \frac{\partial^2 f}{\partial x^2} \\
(f_x)_y = f_{xy} &= \frac{\partial^2 f}{\partial y \partial x} \\
(f_y)_x = f_{yx} &= \frac{\partial^2 f}{\partial x \partial y} \\
(f_y)_y = f_{yy} &= \frac{\partial^2 f}{\partial y^2}
\end{align*}


\begin{theorem}
\textbf{Clairaut's Theorem:} Suppose $f$ is defined on a disk $D$ that contains the 
point $(a, b)$. If $f_{xy}$ and $f_{yx}$ are both continuous on $D$, then: 
\[
f_{xy}(a, b) = f_{yx}(a, b)
\]
\end{theorem}

\section{Tangent Planes and Linear Approximations}

The equation of a \textbf{tangent plane} to the surface $z = f(x, y)$ at the point $P(x_0, y_0, z_0)$ is 
\[
z - z_0 = \parx(x_0, y_0) (x - x_0) + \pary(x_0, y_0) (y - y_0)
\]
if $f$ has continuous partial derivatives.
\\

\textbf{The Linear Approximation of} $f$ at $(a, b)$ is:
\[f(x, y) \approx f(a, b) + \parx(a, b) (x- a) + \pary(a, b) (y-b)\]

Remark: The section in the book about differentiability with epsilon has been omitted for brevity.

\begin{theorem}
If partial derivatives of $f$ exist near $(a, b)$ and are continuous at $(a, b)$ then $f$ is differentiable at $(a, b)$
\end{theorem}

\textbf{Total differential} $dz$ is defined as
\[
dz = \parx dx + \pary dy
\]

\begin{center}
    \includegraphics[width=0.5\textwidth]{images/differential.png}
\end{center}

\section{Chain Rule}

\textbf{Basic chain rule reminder:} $\frac{dy}{dt} = \frac{dy}{dx} \frac{dx}{dt}$
\\

\textbf{Chain Rule Case 1}: If $z = f(x, y)$ is a differentiable function of $x$ and $y$ which are 
both differentiable functions of $t$. Then $z$ is a differentiable function of $t$ and 
\[
\frac{dz}{dt} = \parx \frac{dx}{dt} + \pary \frac{dy}{dt}
\]

\textbf{Chain Rule Case 2}: If $z = f(x, y)$ is a differentiable function of $x$ and $y$ which are 
both differentiable functions of $s$ and $t$. Then 
\[
\frac{\partial z}{\partial s} = \parx \frac{\partial x}{\partial s} + \pary \frac{\partial y}{\partial s} \quad \quad \frac{\partial z}{\partial t} = \parx \frac{\partial x}{\partial t} + \pary \frac{\partial y}{\partial t}
\]

\begin{center}
    \includegraphics[width=0.5\textwidth]{images/chain.png}
\end{center}

\textbf{Chain Rule General Case}: If 
\begin{itemize}
    \item $u$ is a differentiable function of $n$ variables $x_1, \dots, x_n$
    \item Each $x_i$ is a differentiable function of $m$ variables $t_1, \dots, t_m$
\end{itemize}
Then $u$ is a function of $t_1, \dots, t_m$ and 
\[
\frac{\partial u}{\partial t_i} = \frac{\partial u}{\partial x_1} \frac{\partial x_1}{\partial t_i} + \dots + \frac{\partial u}{\partial x_n} \frac{\partial x_n}{\partial t_i}
\]
for all $i = 1, \dots, m$
\\

\textbf{Implicit Differentiation}
\begin{tcolorbox}[colback=blue!5!white, colframe=blue!75!black, title=Formula for Implicit Differentiation (2D)]
Given an equation $F(x, y) = 0$, the derivative of $y$ with respect to $x$ is:
\[
\frac{dy}{dx} = -\frac{\frac{\partial F}{\partial x}}{\frac{\partial F}{\partial y}} = -\frac{F_x}{F_y}
\]
provided that $F_y \neq 0$.
\end{tcolorbox}

\begin{tcolorbox}[colback=blue!5!white, colframe=blue!75!black, title=Formulas for Implicit Differentiation (3D)]
Given an equation $F(x, y, z) = 0$, the partial derivatives of $z$ are:
\[
\frac{\partial z}{\partial x} = -\frac{\frac{\partial F}{\partial x}}{\frac{\partial F}{\partial z}} = -\frac{F_x}{F_z}
\]
\[
\frac{\partial z}{\partial y} = -\frac{\frac{\partial F}{\partial y}}{\frac{\partial F}{\partial z}} = -\frac{F_y}{F_z}
\]
provided that $F_z \neq 0$.
\end{tcolorbox}

\section{Directional Derivatives and the Gradient Vector}

\begin{definition}
If $f$ is a differentiable function of $x, y$, the 
\textbf{directional derivative} of $f$ at $(x_0, y_0)$ in the direction of a 
unit vector $\uu = \langle a, b\rangle$ is 
\[
D_{\uu} f(x, y) = f_x (x, y) a + f_y (x, y) b
\]

\end{definition}

\begin{definition}
If $f$ is a function of $x, y$, then the \textbf{gradient} of $f$ is 
the vector function 
\[\nabla f (x, y) = \langle f_x (x, y), f_y (x, y) \rangle
\]
\end{definition}

With these definitions, we can write the directional derivative cleanly as: 
\[
D_{\uu} f(x, y) = \nabla f(x, y) \cdot \uu\]

Equivalently: 

\[
D_{\uu} f(x, y) = |\nabla f(x, y)| \cos \theta
\]

where $\theta$ os the angle between $\nabla f(x, y)$ and $\uu$

This generalized to three variable case:

\[D_{\uu} f(x, y, z) = \nabla f(x, y, z) \cdot \uu\]

\begin{theorem}
If $f$ is differentiable function of 2 or 3 variables, the max 
value of the directional derivative is $|\nabla f(\xx)|$ and it happens 
when $\uu$ has the same direction as the vector $\nabla f(\xx)$
\end{theorem}

\begin{theorem}
If $f$ is differentiable function of 2 or 3 variables, the min 
value of the directional derivative is $-|\nabla f(\xx)|$ and it happens 
when $\uu$ has the same direction as the vector $-\nabla f(\xx)$
\end{theorem}

\textbf{Equation of tangent plane} to level surface $F(x, y, z) = k$ at 
$P(x_0, y_0, z_0)$ is:

\[F_x(x_0, y_0, z_0) (x - x_0) + F_y(x_0, y_0, z_0) (y - y_0) + F_z(x_0, y_0, z_0) (z - z_0) = 0\]

\textbf{Equation of normal line} to level surface $F(x, y, z) = k$ at 
$P(x_0, y_0, z_0)$ is:

\[
\frac{x-x_0}{F_x} = \frac{y-y_0}{F_y} = \frac{z-z_0}{F_z}
\]

$\nabla f(\xx)$ is perpedicular to the level curve or level surface of $f$
through $\xx$

\section{Maximum and Minimum Values}

\begin{definition}
A point $(a, b)$ is called a \textbf{critical point} of $f$ if 
$f_x(a, b) = 0$ and $f_y(a, b) = 0$, or if one of these partial derivatives 
does not exist.
\end{definition}

\begin{theorem}
If $f$ has a local max or min at $(a, b)$ and first order partial derivatives
exist there then $f_x(a, b) = 0 $ and $f_y(a, b) = 0 $
\end{theorem}

\textbf{Second Derivative Test:}

Let $D(a, b) = f_{xx}(a, b) f_{yy}(a, b) - [f_{xy}(a, b)]^2$
\begin{itemize}
    \item If $D(a, b) > 0$ and $f_{xx}(a, b) > 0$, then $(a, b)$ is a local min
    \item If $D(a, b) > 0$ and $f_{xx}(a, b) < 0$, then $(a, b)$ is a local max
    \item If $D(a, b) < 0$, then $(a, b)$ is a saddle point 
    \item If $D(a, b) = 0$ then the test gives no information
\end{itemize}

If you are crazy then you can use the following to remember the formula for $D$:

\[
D = \begin{vmatrix}
f_{xx} & f_{xy}\\
f_{yx} & f_{yy} \\
\end{vmatrix} = f_{xx} f_{yy} - (f_{xy})^2
\]

\begin{definition}
$f(a, b)$ is \textbf{absolute maximum} of $f$ on $D$ if $f(a, b) \ge f(x, y)$ 
for all $(x, y)$ in $D$

$f(a, b)$ is \textbf{absolute minimum} of $f$ on $D$ if $f(a, b) \le f(x, y)$ 
for all $(x, y)$ in $D$
\end{definition}

\begin{theorem}
Extreme Value Theorem: A continuous function $f$ will attain both an 
absolute minimum and an absolute maximum somewhere on a bounded, 
closed set in $\mathbb{R}^2$
\end{theorem}

To find where this happens:
\begin{enumerate}
    \item Find the values of $f$ at the critical points of $f$ in $D$
    \item Find the extreme values of $f$ on the boundary of $D$
    \item The largest from these two is absolute max, the smallest 
    is absolute min
\end{enumerate}

\section{Lagrange Multipliers}

\textbf{The Method of Lagrange Multipliers}

To find the minimum or maximum values of $f(x, y, z)$ subject to the constraint  
$g(x, y, z) = k$:

\begin{itemize}
    \item Find all values of $x, y, z, \lambda$ such that
    \[
        \nabla f(x, y, z) = \lambda \, \nabla g(x, y, z)
    \]
    and
    \[
        g(x, y, z) = k.
    \]
    \item Evaluate $f$ at all points obtained above.  
    The largest value gives the maximum; the smallest gives the minimum.
\end{itemize}

The equations in the first step can be written componentwise as:
\[
    f_x = \lambda g_x, \qquad 
    f_y = \lambda g_y, \qquad
    f_z = \lambda g_z, \qquad
    g(x, y, z) = k.
\]

\textbf{Two Constraints}

If we want to find extrema subject to two constraints  
$g(x, y, z) = k$ and $h(x, y, z) = c$, we solve:
\[
    \nabla f(x, y, z) 
    = \lambda \nabla g(x, y, z) 
    + \mu \nabla h(x, y, z),
\]
together with
\[
    g(x, y, z) = k, \qquad h(x, y, z) = c.
\]

Equivalently, this system is:
\[
\begin{aligned}
    f_x &= \lambda g_x + \mu h_x, \\
    f_y &= \lambda g_y + \mu h_y, \\
    f_z &= \lambda g_z + \mu h_z, \\
    g(x, y, z) &= k, \\
    h(x, y, z) &= c.
\end{aligned}
\]


\chapter{Multiple Integrals}

\section{Double Integrals over Rectangles}

\textbf{Volumes and Double Integrals}

If $f(x, y) \ge 0$, then the volume $V$ of solid that lies above the rectangle 
$R$ and below surface $z = f(x, y)$ is 
\[
V = \iint_R f(x,y)\, dA
\]

\textbf{Midpoint Rule for Double Integrals}:

\[\iint_D f(x,y)\, dA \approx \sum_{i=1}^{m} \sum_{j=1}^{n} f(\bar{x_i}, \bar{y_j}) \Delta A\]

where $\bar{x_i}$ is the midpoint of $[x_{i-1}, x_i]$, $\bar{y_j}$ is the 
midpoint of $[y_{j-1}, y_j]$

\textbf{Fubini's Theorem.}: If $f$ is continuous on a rectangle $R = \{(x, y) | a \le x \le b, c \le y \le d\}$
then 

\[
\iint_R f(x, y)\, dA = \int_{a}^{b} \int_{c}^{d} f(x, y) \, dy \, dx 
= \int_{c}^{d} \int_{a}^{b} f(x, y) \, dx \, dy 
\]

If $f(x, y)$ can be written in the form $f(x, y) = g(x)h(y)$ then 

\[
\iint_R f(x, y)\, dA = \iint_R g(x)h(y) \, dA = 
\int_{a}^{b} g(x) \, dx \int_{c}^{d} h(y) \, dy
\] where $R = [a, b] \times [c, d]$

\textbf{Average value} of a function $f$ on a rectangle $R$ is:

\[
f_{\text{average}} = \frac{1}{\text{Area}(R)} \iint_R f(x, y)\, dA
\]

Interesting: if $f(x, y) \ge 0$, then $\text{Area}(R) \times f_{\text{average}} = \iint_R f(x, y)\, dA$

This says that the box with base $R$ and height $f_{\text{average}}$ has the 
same volume as the solid that lies under the graph of $f$. 

\section{Double Integrals over General Regions}

A plane region $D$ is \textbf{type} \Romannum{1} if it lies between the graphs of two 
continuous functions of $x$, i.e.
\[D = \{(x, y) | a \le x \le b, g_1(x) \le y \le g_2(x)\}\]

In this case, we have:

\[
\iint_{D} f(x, y) \, dA = \int_{a}^{b} \int_{g_1(x)}^{g_2(x)} f(x, y) \, dy \, dx
\]

Illustration (think of drawing a vertical arrow across the region):

\begin{center}
    \includegraphics[width=0.5\textwidth]{images/type1.png}
\end{center}

A plane region $D$ is \textbf{type} \Romannum{2} if it lies between the graphs of two 
continuous functions of $y$, i.e.
\[D = \{(x, y) | c \le y \le d, h_1(y) \le x \le h_2(y)\}\]

In this case, we have:

\[
\iint_{D} f(x, y) \, dA = \int_{c}^{d} \int_{h_1(y)}^{h_2(y)} f(x, y) \, dx \, dy
\]

Illustration (think of drawing a horizontal arrow across the region):

\begin{center}
    \includegraphics[width=0.5\textwidth]{images/type2.png}
\end{center}

\textbf{Properties:}
\begin{itemize}
    \item $\iint_D [f(x,y) + g(x,y)] \, dA = \iint_D f(x,y) \, dA + \iint_D g(x,y) \, dA$
    \item $\iint_D c f(x,y) \, dA = c \iint_R f(x,y) \, dA$
    \item If $f(x,y) \ge g(x,y)$ for all $(x,y)$ in $D$, then $\iint_D f \, dA \ge \iint_D g \, dA$
    \item If $D = D_1 \cup D_2$ where $D_1, D_2$ don't overlap except at boundaries,
    then 
    
    $\iint_D f(x, y) \, dA = \iint_{D_1} f(x, y) \, dA + \iint_{D_2} f(x, y) \, dA$
    \item $\iint_D 1 \, dA = \text{Area}(D)$
\end{itemize} 

\textbf{Strategy: Switching the Order of Integration}
Sometimes an integral is impossible to evaluate in the given order. 
To solve this:
\begin{enumerate}
    \item Sketch the region $D$ based on the given bounds.
    \item Interpret the region as the \textit{other} Type (switch from Type I to II or vice versa).
    \item Rewrite the integral with the new bounds and new order ($dx \, dy \leftrightarrow dy \, dx$).
\end{enumerate}

\section{Double Integrals in Polar Coordinates}

\textbf{Relationship between rectangular coordinates and polar coordinates:}

\[
r^2 = x^2 + y^2, \quad x = r \cos \theta, \quad y = r \sin \theta
\]

\textbf{Double Integrals in Polar Coordinates}

Polar Rectangle: 
$R = \{(r, \theta) \, | \, a \le r \le b, \alpha \le \theta \le \beta\}$

\begin{theorem} 
If $f$ is continuous on a polar rectangle $R$ where $0 \le a \le r \le b$,
$\alpha \le \theta \le \beta$ and $0 \le \beta - \alpha \le 2 \pi$,
then 
\[
\iint_{R} f(x, y) \, dA = \int_{\alpha}^{\beta} \int_{a}^{b} 
f(r \cos \theta, r \sin \theta) \, r \, dr \, d \theta
\]
\end{theorem}

More complicated region: 
$D = \{(r, \theta) \, | \, \alpha \le \theta \le \beta, h_1(\theta) \le r \le h_2(\theta)\}$
then the following is true:

\[
\iint_{D} f(x, y) \, dA = \int_{\alpha}^{\beta} \int_{h_1(\theta)}^{h_2(\theta)} 
f(r \cos \theta, r \sin \theta) \, r \, dr \, d \theta
\]

\textbf{Common Polar Bounds:}
\begin{itemize}
    \item \textbf{Full Circle} (radius $a$): $0 \le r \le a$, $0 \le \theta \le 2\pi$
    \item \textbf{Top Semicircle}: $0 \le r \le a$, $0 \le \theta \le \pi$
    \item \textbf{Right Semicircle}: $0 \le r \le a$, $-\pi/2 \le \theta \le \pi/2$
    \item \textbf{Annulus (Ring)} (radii $a$ and $b$): $a \le r \le b$, $0 \le \theta \le 2\pi$
\end{itemize}

\section{Applications of Double Integrals (Probability)}

A function $f(x, y)$ is a \textbf{joint density function} for a pair of random variables $(X, Y)$ if:
\begin{itemize}
    \item $f(x, y) \ge 0$ for all $(x, y)$
    \item The total volume under the graph is 1:
    \[ \iint_{\mathbb{R}^2} f(x, y) \, dA = \int_{-\infty}^{\infty} \int_{-\infty}^{\infty} f(x, y) \, dx \, dy = 1 \]
\end{itemize}

The probability that $(X, Y)$ lies in a region $D$ is the volume under the surface over $D$:
\[ P((X, Y) \in D) = \iint_D f(x, y) \, dA \]

\begin{itemize}
    \item \textbf{Mean of X:} 
    \[ E[X] = \iint_{\mathbb{R}^2} x f(x, y) \, dA \]
    \item \textbf{Mean of Y:} 
    \[ E[Y] = \iint_{\mathbb{R}^2} y f(x, y) \, dA \]
\end{itemize}

Two random variables $X$ and $Y$ are \textbf{independent} if their joint density function is the product of their individual p.d.f.'s:
\[ f(x, y) = f_1(x) f_2(y) \]
Where $f_1$ and $f_2$ are the single-variable density functions for $X$ and $Y$.

\textbf{Exponential Random Variables}

If $X$ and $Y$ are independent exponential variables with means $\mu_1$ and $\mu_2$:
\[
f(x, y) = \begin{cases}
    \frac{1}{\mu_1 \mu_2} e^{-x/\mu_1} e^{-y/\mu_2} & \text{if } x \ge 0, y \ge 0 \\
    0 & \text{otherwise}
\end{cases}
\]

The single variable form is $f(t) = \frac{1}{\mu}e^{-t/\mu}$ for $t \ge 0$.

\section{Surface Area}

The area of the surface with equation $z = f(x, y)$, $(x, y) \in D$, where 
$f_x, f_y$ are continuous, is 

\[
A(S) = \iint_{D} \sqrt{[f_x(x, y)]^2 + [f_y(x, y)]^2 + 1} \; dA
\]

\section{Triple Integrals}

If $f$ is continuous on the rectangular box $B = [a, b] \times [c, d] \times [r, s]$
then 
\[
\iiint_B f(x, y, z) dV = \int_{r}^{s} \int_{c}^{d} \int_{a}^{b} f(x, y, z) \; dx \; dy \; dz
\]

A region $E$ is \textbf{type 1} if it lies between two
continuous functions of $x, y$

\[
E = \{(x, y, z) | (x, y) \in D, u_1(x, y) \le z \le u_2(x, y)\}
\]

where $D$ is the projection of $E$ onto $xy$-plane. 

\begin{center}
    \includegraphics[width=0.4\textwidth]{images/type1_triple.png}
\end{center}

In this case we have:

\[
\iiint_E f(x, y, z) dV = \iint_D \left[ \int_{u_1(x, y)}^{u_2(x, y)} f(x, y, z) \; dz \right] \; dA
\]

Then we evaluate it based on whether the projected region $D$ is type \Romannum{1} or 
\Romannum{2}

\begin{itemize}
    \item If $D$ is type \Romannum{1} then 
    $E = \{(x, y, z) | a \le x \le b, g_1(x) \le y \le g_2(x), u_1(x, y) \le z \le u_2(x, y)\}$

    so the integral becomes:

    \[
        \iiint_E f(x, y, z) dV = \int_{a}^{b} \int_{g_1(x)}^{g_2(x)} \int_{u_1(x, y)}^{u_2(x, y)} f(x, y, z) \; dz \; dy \; dx
    \]

    \item If $D$ is type \Romannum{2} then 
    $E = \{(x, y, z) | c \le y \le d, h_1(y) \le x \le h_2(y), u_1(x, y) \le z \le u_2(x, y)\}$

    so the integral becomes:

    \[
    \iiint_E f(x, y, z) dV = \int_{c}^{d} \int_{h_1(y)}^{h_2(y)} \int_{u_1(x, y)}^{u_2(x, y)} f(x, y, z) \; dz \; dx \; dy
    \]
\end{itemize}

There is also \textbf{type 2}: $E = \{(x, y, z) | (y, z) \in D, u_1(y, z) \le x \le u_2(y, z)\}$
where $D$ is the projection of $E$ onto the $yz$-plane.

\begin{center}
    \includegraphics[width=0.4\textwidth]{images/type2_triple.png}
\end{center}

And finally \textbf{type 3}: $E = \{(x, y, z) | (x, z) \in D, u_1(x, z) \le y \le u_2(x, z)\}$
where $D$ is the projection of $E$ onto the $xz$-plane.

\begin{center}
    \includegraphics[width=0.4\textwidth]{images/type3_triple.png}
\end{center}

Evaluation generalizes from the previous case.

Changing the Order of Integration also generalizes from previous.

Interesting: if $f(x, y, z) = 1$ for all points in $E$ 
then $V(E) = \iiint_E dV$ is in fact the volume of $E$. Wow. 


\section{Triple Integrals in Cylindrical Coordinates}

A point $P$ in the \textbf{cylindrical coordinate system} is represented by a 
triple $(r, \theta, z)$ where $r, \theta$ are polar coordinates of the projection of 
$P$ onto the $xy$-plane and $z$ is the directed distance from the $xy$-plane to $P$.

\begin{center}
    \includegraphics[width=0.4\textwidth]{images/cylindrical.png}
\end{center}

To convert from cylindrical to rectangular coordinates, we use the equations:

\[
x = r \cos \theta \quad \quad y = r \sin \theta \quad \quad z = z
\]

To convert back to cylindrical, we use:

\[
r^2 = x^2 + y^2 \quad \quad \tan \theta = \frac{y}{x} \quad \quad z = z
\]

\textbf{Triple Integrals in Cylindrical Coordinates}:

Suppose that $E$ is a region whose projection $D$ on the $xy$-plane
is conveniently described in polar coordinates (e.g., a polar rectangle):

\[
E = \{(r, \theta, z) \mid \alpha \le \theta \le \beta, h_1(\theta) \le r \le h_2(\theta), u_1(r, \theta) \le z \le u_2(r, \theta) \}
\]

The fundamental identity for the volume element is:
\[
dV = r \, dz \, dr \, d\theta
\]

Thus, the formula for triple integration is:

\[
\iiint_E f(x, y, z) \, dV = \int_{\alpha}^{\beta} \int_{h_1(\theta)}^{h_2(\theta)} \int_{u_1(r \cos \theta, r \sin \theta)}^{u_2(r \cos \theta, r \sin \theta)} f(r \cos \theta, r \sin \theta, z) \; r \; dz \, dr \, d\theta
\]

\textbf{Note:} It is crucial to remember the factor of $\mathbf{r}$ in the integrand. 
This comes from the area element in polar coordinates ($dA = r \, dr \, d\theta$).

Use cylindrical coordinates when:
\begin{enumerate}
    \item The domain $E$ has symmetry about the $z$-axis (e.g., cylinders, cones).
    \item The integrand involves the expression $x^2 + y^2$.
\end{enumerate}

\section{Triple Integrals in Spherical Coordinates}

The \textbf{spherical coordinates} of a point $P$ are $(\rho, \theta, \phi)$, where:
\begin{itemize}
    \item $\rho$ is the distance from the origin to $P$ ($\rho \ge 0$).
    \item $\theta$ is the same angle as in cylindrical coordinates ($0 \le \theta \le 2\pi$).
    \item $\phi$ is the angle between the positive $z$-axis and the line segment $OP$ ($0 \le \phi \le \pi$).
\end{itemize}

\begin{center}
    \includegraphics[width=0.4\textwidth]{images/spherical.png}
\end{center}


\textbf{Converting to Rectangular:}

\[
x = \rho \sin \phi \cos \theta \quad \quad y = \rho \sin \phi \sin \theta 
\quad \quad z = \rho \cos \phi
\]

\textbf{Converting to Spherical:}

\[
\rho^2 = x^2 + y^2 + z^2
\]

\textbf{Integration Volume Element:}

The volume element in spherical coordinates is given by:
\[
dV = \rho^2 \sin \phi \, d\rho \, d\theta \, d\phi
\]



\textbf{Integrating over a Spherical Wedge:}

A "spherical wedge" is the simplest region, defined by constant bounds (similar to a rectangular box):

\[
E = \{(\rho, \theta, \phi) \mid a \le \rho \le b, \alpha \le \theta \le \beta, c \le \phi \le d\}
\]

For this region, the integral becomes:

\[
\iiint_E f(x, y, z) \; dV = \int_{c}^{d} \int_{\alpha}^{\beta} \int_{a}^{b}
f(\rho \sin \phi \cos \theta, \rho \sin \phi \sin \theta, \rho \cos \phi) 
\; \rho^2 \sin \phi \; d\rho \, d\theta \, d\phi
\]

\end{document}